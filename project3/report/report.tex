% \documentclass[10pt, a4paper]{amsart}
\documentclass[%
reprint,
nofootinbib,
%superscriptaddress,
%groupedaddress,
%unsortedaddress,
%runinaddress,
%frontmatterverbose,
%preprint,
%showpacs,preprintnumbers,
%nofootinbib,
%nobibnotes,
%bibnotes,
amsmath,amssymb,
aps,
%pra,
%prb,
%rmp,
%prstab,
%prstper,
%floatfix,
]{revtex4-1}
\usepackage{preamble}
\usepackage[]{graphicx}
\usepackage[]{hyperref}
\usepackage[]{physics}
\usepackage[]{listings}
\usepackage[T1]{fontenc}
\usepackage{color}
\usepackage[]{subcaption}
\usepackage[ruled,vlined]{algorithm2e}

\definecolor{mygreen}{rgb}{0,0.6,0}
\definecolor{mymauve}{rgb}{0.58,0,0.82}

\newcommand\todo[1]{\textcolor{red}{#1}}

\lstset{ %
	backgroundcolor=\color{white},   % choose the background color; you must add \usepackage{color} or \usepackage{xcolor}
	basicstyle=\footnotesize,        % the size of the fonts that are used for the code
	breakatwhitespace=false,         % sets if automatic breaks should only happen at whitespace
	breaklines=true,                 % sets automatic line breaking
	captionpos=b,                    % sets the caption-position to bottom
	commentstyle=\color{mygreen},    % comment style
	deletekeywords={...},            % if you want to delete keywords from the given language
	escapeinside={\%*}{*)},          % if you want to add LaTeX within your code
	extendedchars=true,              % lets you use non-ASCII characters; for 8-bits encodings only, does not work with UTF-8
	frame=single,	                   % adds a frame around the code
	keepspaces=true,                 % keeps spaces in text, useful for keeping indentation of code (possibly needs columns=flexible)
	keywordstyle=\color{blue},       % keyword style
	language=c++,                    % the language of the code
	otherkeywords={*,...},           % if you want to add more keywords to the set
	rulecolor=\color{black},         % if not set, the frame-color may be changed on line-breaks within not-black text (e.g. comments (green here))
	showspaces=false,                % show spaces everywhere adding particular underscores; it overrides 'showstringspaces'
	showstringspaces=false,          % underline spaces within strings only
	showtabs=false,                  % show tabs within strings adding particular underscores
	stepnumber=2,                    % the step between two line-numbers. If it's 1, each line will be numbered
	stringstyle=\color{mymauve},     % string literal style
	tabsize=2,	                     % sets default tabsize to 2 spaces
}


\begin{document}
	
\title{Ordinary differential equations\\
	\normalsize{Building a model for the solar system} \\
	\hrulefill\small{ FYS3150: Computational Physics }\hrulefill}

\author{Sigurd Sandvoll Sundberg}\homepage{https://github.com/SigurdSundberg/FYS3150/tree/master/project3}

\affiliation{%
	Department of Geosciences, University of Oslo\\
}%

\date{\today}

\begin{abstract}%[0]

\end{abstract}

\maketitle 

\section{Introduction} %[0]

\section{Conclusion} %[0]
\appendix

%%% footnote
% rainforest\footnote{Writing out a general case will also take up more paper
% space}.

%%% Matrix with line through between last elements
% \begin{equation}
% \left[
% \begin{array}{cccc|c}
% 1 & c_1/\beta_1 & 0 & 0 & \tilde{f}_1 \\
% 0 & 1 & c_2/\beta_2 & 0  & \tilde{f}_2 \\
% 0 & 0 & 1 & c_3/\beta_3 & \tilde{f}_3 \\
% 0 & 0 & 0 & 1 & \tilde{f}_4
% \end{array}
% \right] \sim
% \left[
% \begin{array}{cccc|c}
% 1 & c_1/\beta_1 & 0 & 0 & \tilde{f}_1 \\
% 0 & 1 & c_2/\beta_2 & 0  & \tilde{f}_2 \\
% 0 & 0 & 1 & 0 & \tilde{f}_3 -\frac{c_3}{\beta_3}\tilde{f}_4 \\
% 0 & 0 & 0 & 1 & \tilde{f}_4
% \end{array}
% \right]
% \end{equation}

%%% Listing
% \lstinputlisting[language=c++, firstline=146,
% lastline=158]{../problems.cpp}

%%% LU matrix, with diag dots for general case
% \begin{equation}
% A = LU =
% \begin{bmatrix}
% 1 & 0 & 0 & \dots & 0 & 0 \\
% l_{21} & 1 & 0 & \dots & 0 & 0 \\
% l_{31} & l_{32} & 1 & \dots & 0 & 0 \\
%   &\vdots & & \ddots & \vdots  & \\
% l_{n-11} & l_{n-12} & l_{n-13} & \dots & 1 & 0 \\
% l_{n1} & l_{n2} & l_{n3} & \dots & l_{nn-1} & 1
% \end{bmatrix}
% \begin{bmatrix}
% u_{11} & u_{12} & u_{13} & \dots & u_{1n-1} & u_{1n} \\
% 0 & u_{22} & u_{23} & \dots & u_{2n-1} & u_{2n} \\
% 0 & 0 & u_{33} & \dots & u_{3n-1} & u_{3n} \\
%   &\vdots & & \ddots & \vdots  & \\
% 0 & 0 & 0 & \dots & u_{n-1n-1} & u_{n-1n} \\
% 0 & 0 & 0 & \dots & 0 & u_{nn}
% \end{bmatrix}
% \end{equation}

%%% Table
% \begin{table}[h]
% \caption{Elapsed time for increasing $n$}
% \begin{tabular}{lcc}
% \hline
% n & TDMA [s] & LU [s] \\ \hline
% 10 & 0.0000035 & 0.00106083 \\
% 100 & 0.0000116 & 0.0022319 \\
% 1000 & 0.000077892 & 0.0677764 \\
% 10000 & 0.000878769 & 21.9247 \\
% 100000 & 0.00757418 & n/a \\
% 1000000 & 0.08616075 & n/a \\
% 10000000 & 0.76534 & n/a
% \end{tabular}
% \label{tab:solver_times}
% \end{table}

%%% Figure
% \begin{figure}[h]
  % \centering
  % \includegraphics[width=0.9\linewidth]{figures/relerror.png}
  % \caption{Plot of maximum relative error as a function of step size}
  % \label{fig:relerror}
% \end{figure}

%%% Cition and bilbliography%%
% # \emph{Thomas Algorithm} \cite{thomasalgo} How to cite.
% \begin{thebibliography}{10}
  % \bibitem{thomasalgo}{Thomas, L.H. (1949), \emph{Elliptic
        % Problems in Linear Differential Equations over a
        % Network}. Watson Sci. Comput. Lab Report, Columbia University,
      % New York.}
      % \bibitem{morten}{Hjorth-Jensen, M. (2015). \emph{Computational
        % Physics - Lecture Notes 2015}. University of Oslo}
    % \bibitem{golub}{Golub, G.H., van Loan, C.F. (1996). \emph{Matrix
          % Computations} (3rd ed.), Baltimore: John Hopkins.}
% \end{thebibliography}

\bibliography{bib_proj3}
\bibliographystyle{plain}


\end{document}
