% \documentclass[10pt, a4paper]{amsart}
\documentclass[%
reprint,
nofootinbib,
%superscriptaddress,
%groupedaddress,
%unsortedaddress,
%runinaddress,
%frontmatterverbose,
%preprint,
%showpacs,preprintnumbers,
%nofootinbib,
%nobibnotes,
%bibnotes,
amsmath,amssymb,
aps,
%pra,
%prb,
%rmp,
%prstab,
%prstper,
%floatfix,
]{revtex4-1}
\usepackage[]{graphicx}
\usepackage{float}
\usepackage[]{hyperref}
\usepackage[]{physics}
\usepackage[]{listings}
\usepackage[T1]{fontenc}
\usepackage{color}
\usepackage[]{subcaption}
\usepackage[ruled,vlined]{algorithm2e}
\usepackage{amssymb, amsmath}

\definecolor{mygreen}{rgb}{0,0.6,0}
\definecolor{mymauve}{rgb}{0.58,0,0.82}

\newcommand\todo[1]{\textcolor{red}{#1}}

\lstset{ %
	backgroundcolor=\color{white},   % choose the background color; you must add \usepackage{color} or \usepackage{xcolor}
	basicstyle=\footnotesize,        % the size of the fonts that are used for the code
	breakatwhitespace=false,         % sets if automatic breaks should only happen at whitespace
	breaklines=true,                 % sets automatic line breaking
	captionpos=b,                    % sets the caption-position to bottom
	commentstyle=\color{mygreen},    % comment style
	deletekeywords={...},            % if you want to delete keywords from the given language
	escapeinside={\%*}{*)},          % if you want to add LaTeX within your code
	extendedchars=true,              % lets you use non-ASCII characters; for 8-bits encodings only, does not work with UTF-8
	frame=single,	                   % adds a frame around the code
	keepspaces=true,                 % keeps spaces in text, useful for keeping indentation of code (possibly needs columns=flexible)
	keywordstyle=\color{blue},       % keyword style
	language=c++,                    % the language of the code
	otherkeywords={*,...},           % if you want to add more keywords to the set
	rulecolor=\color{black},         % if not set, the frame-color may be changed on line-breaks within not-black text (e.g. comments (green here))
	showspaces=false,                % show spaces everywhere adding particular underscores; it overrides 'showstringspaces'
	showstringspaces=false,          % underline spaces within strings only
	showtabs=false,                  % show tabs within strings adding particular underscores
	stepnumber=2,                    % the step between two line-numbers. If it's 1, each line will be numbered
	stringstyle=\color{mymauve},     % string literal style
	tabsize=2,	                     % sets default tabsize to 2 spaces
}


\begin{document}
	
\title{Monte Carlo simulations\\
	\normalsize{The Ising Model and Phase transition} \\
	\hrulefill\small{ FYS3150: Computational Physics }\hrulefill}

\author{Sigurd Sandvoll Sundberg}\homepage{https://github.com/SigurdSundberg/FYS3150}

\affiliation{%
	Department of Geosciences, University of Oslo\\
}%

\date{\today}

\begin{abstract}%[0]

\end{abstract}

\maketitle 

\section{Introduction} %0

\section{Theory} %0
The Ising model is a binary system, where the objects at each lattice site can take one of two values, either spin up or spin down. 
The energy of the Ising model, with no external magnetic field can be expressed as 
\begin{equation}
	E = -J\sum_{\langle k,e \rangle} s_ks_e,
\end{equation}
where we sum over the nearest neighbors interactions at each lattice site and $s_i$ can take the values $\pm 1$, which represents either spin up or spin down, respectively. 

\subsection{Canonical Ensemble}
In this article we will be working with an canonical ensemble, where the energy follows an expectation value at a given temperature. For us to calculate the expectation values such as the mean energy $\langle E \rangle$, we will be using a probability distribution, namely the Boltzmann distribution
\begin{equation}
	P_i(\beta) = \frac{e^{-\beta E_i}}{Z}
\end{equation}
where $\beta = 1/k_BT$ being the inverse temperature, $k_B$ is the Boltzmann constant, $E_i$ is the energy at microstate $i$ and Z is the partition function. We will in this article use natural units, that is $k_B = J = 1$. 

The partition function Z is given as 
\begin{equation}\label{eq:Z}
	Z = \sum_{i=1}^{M}e^{-\beta E_i},
\end{equation} 
where we sum over all microstates M.

The expectation value of the energy can be calculated from the probability distribution $P_i$ as 
\begin{equation}\label{eq:E}
	\langle E\rangle = \sum_{i=1}^{M}E_iP_i(\beta) = \frac{1}{Z}\sum_{i=1}^ME_ie^{-\beta E_i}
\end{equation}

The corresponding variance is defined as 
\begin{equation}
	\begin{split}
\sigma_E^2 &= \langle E^2 \rangle -\langle E\rangle^2 \\
&= \frac{1}{Z}\sum_{i=1}^ME_i^2e^{-\beta E_i} - \left(\frac{1}{Z}\sum_{i=1}^ME_i^2e^{-\beta E_i}\right)^2
	\end{split}
\end{equation}
If we divide by $k_BT^2$, we obtain the specific heat 
\begin{equation}\label{eq:cv}
	C_v = \frac{1}{k_BT^2}\left(\langle E^2 \rangle -\langle E\rangle^2\right).
\end{equation}

In the same way we can evaluate the mean magnetization through 
\begin{equation}\label{eq:M}
	\langle\mathcal{M}\rangle = \sum_i^M\mathcal{M}P_i(\beta) = \frac{1}{Z}\sum_i^M\mathcal{M}e^{-\beta E_i}
\end{equation}
and the corresponding variance 
\begin{equation}
\begin{split} 
		\sigma_\mathcal{M}^2			&=\langle \mathcal{M}^2 \rangle -\langle \mathcal{M}\rangle^2 \\
			&= \frac{1}{Z}\sum_{i=1}^M\mathcal{M}_i^2e^{-\beta E_i} - \left(\frac{1}{Z}\sum_{i=1}^M\mathcal{M}_i^2e^{-\beta E_i}\right)^2.
		\end{split} 
\end{equation}
This quantity defines the susceptibility $\mathcal{X}$
\begin{equation}\label{eq:x}
	\mathcal{X} = \frac{1}{k_BT}\left(\langle \mathcal{M}^2 \rangle -\langle \mathcal{M}\rangle^2\right).
\end{equation}
\section{Phase Transition}
\subsection{Analytical Results}
In 1944 Lars Onsager was the first to solve the Ising model analytically for a general lattice size L\cite{LarsOnsager}. He found that the critical temperature of the system is 
\begin{equation}
	T_C = \frac{2J}{k_Bln(1+\sqrt{2})}\simeq 2.269\dots
\end{equation} 
We will use the ideas from phase transition to try to estimate $T_C$, using larger lattice size. 

However first we need a benchmark for our code, where we will use a 2x2 system with periodic boundary conditions to calculate numerical expectation values. The analytical expectation values and the partition function can be calculated through equations (\ref{eq:E}, \ref{eq:Z}, \ref{eq:M}), and the value displayed in table \ref{tab:2x2}. Likewise we can find values of $E^2$ and $M^2$. 

We find that 
\begin{align}
	Z &= 12 + 4\cosh(8J\beta) && \beta = (k_BT)^{-1}\\
	\langle E\rangle &= -\frac{32J}{Z}\sinh(8J\beta)\\
	\langle E^2\rangle &= \frac{256}{Z}\cosh(8J\beta)\\
	\langle M\rangle &= 0\\
	\langle M^2\rangle &=\frac{32}{Z}\left(e^{8J\beta} + 1\right)\\
	\langle \abs{M}\rangle &= \frac{8}{Z}\left(e^{8J\beta} + 2\right)
\end{align}

From equation \eqref{eq:cv}, the heat capacity of the system is 
\begin{equation}
	C_v = \frac{256}{Zk_BT^2}\left( \cosh(8J\beta) - \frac{4}{Z}\sinh^2(8J\beta)\right)
\end{equation}
and we can find the susceptibility from equation \eqref{eq:x}, which can be expressed as 
\begin{equation}
	\mathcal{X} = \frac{32\beta}{Z}\left(e^{8J\beta} +1\right).
\end{equation}
\begin{table}
	\caption{Display over the possible spin configurations for the 2x2 lattice. As well as listing the energy and magnetization of system.}
	\label{tab:2x2}
\begin{tabular}{|c|c|c|c|c|}
	\hline 
	$N_\uparrow$ & Degeneracy & E  & M & Configurations\\
	\hline
	4 & 1 & -8J & 4 & $\begin{bmatrix}\uparrow & \uparrow \\ \uparrow&\uparrow\end{bmatrix}$\\
	3 & 4 & 0 & 2 & $\begin{bmatrix}\downarrow & \uparrow \\ \uparrow&\uparrow\end{bmatrix}\begin{bmatrix}\uparrow & \downarrow \\ \uparrow&\uparrow\end{bmatrix}\begin{bmatrix}\uparrow & \uparrow \\ \downarrow&\uparrow\end{bmatrix}\begin{bmatrix}\uparrow & \uparrow \\ \uparrow&\downarrow\end{bmatrix}$\\
	2 & 4 & 0 & 0 &$\begin{bmatrix}\downarrow & \downarrow \\ \uparrow&\uparrow\end{bmatrix}\begin{bmatrix}\uparrow & \uparrow \\ \downarrow&\downarrow\end{bmatrix}\begin{bmatrix}\downarrow & \uparrow \\ \downarrow&\uparrow\end{bmatrix}\begin{bmatrix}\uparrow & \downarrow \\ \uparrow&\downarrow\end{bmatrix}$\\
	2 & 2 & 8J & 0&$\begin{bmatrix}\downarrow & \uparrow \\ \uparrow&\downarrow\end{bmatrix}\begin{bmatrix}\uparrow & \downarrow \\ \downarrow&\uparrow\end{bmatrix}$\\
	1 & 4 & 0 & -2&$\begin{bmatrix}\uparrow & \downarrow \\ \downarrow&\downarrow\end{bmatrix}\begin{bmatrix}\downarrow & \uparrow \\ \downarrow&\downarrow\end{bmatrix}\begin{bmatrix}\downarrow & \downarrow \\ \uparrow&\downarrow\end{bmatrix}\begin{bmatrix}\downarrow & \downarrow \\ \downarrow&\uparrow\end{bmatrix}$\\
	0 & 1 & -8J & -4&$\begin{bmatrix}\downarrow & \downarrow \\ \downarrow&\downarrow\end{bmatrix}$\\
	\hline
\end{tabular}
\end{table}




\section{Algorithms}%0
\section{Periodic Boundary Conditions}
When applying periodic boundary conditions we need a way to index ghosts points. In this project to represent our lattice we will use a matrix of size LxL, thus needing to index points outside the scope of our lattice. The simple way of doing so would be to implement \texttt{if}-statements to check whether we are indexing a ghost point or not. This is however slow and hard to optimize. 

A second possibility for handling the ghost points would be to expand our matrix to a (L+2)x(L+2) matrix, where we have the ability to index the ghosts points directly without the use of \texttt{if}-statements. However this runs into its own limitations, namely we would need to update the ghosts points whenever we change its corresponding value. In a 1-dimensional case this would lead to us having to update index 0 if we made a change to index n, for an array from 0 to n+1. This can be implemented using either \texttt{if}-statements or using properties of modulus operator. 

A third method, which is the one used in this project, is using the modulus operator directly on a LxL matrix. This is done by looking at the point of interest in the matrix and compute its neighboring indexes by the following equation
\begin{equation}
	\text{neighbor} = (\text{current} + \text{L} + \text{position}) \% (\text{L})
\end{equation}
where current is the current matrix index, L is the dimensionality of the matrix and position is the index of the neighbor relative to current position. 
This equation handles both the ghost points and the normal points of the matrix and creates an efficient way for us to access all points of the matrix. 

By implementing this as an inline function, we are able to, at compile time, to reduce the time used of our program. 


\section{Metropolis-Hastings algorithm}
\begin{algorithm}[H]\label{algo:metropolis}
	\SetAlgoLined
	\caption{Monte Carlo simulation with Metropolis sampling}
	\texttt{Pick T and $L_i$}	\\
	\texttt{Compute $E_i$ and $M_i$}\\
	\For{\texttt{i = 0; i < MC cycles; i++}}{
		\For{\texttt{j = 0; j < Total Spins; j++}}{
			\texttt{Sample a random index of the spin matrix}\\
			\texttt{Compute $\Delta E$}\\
			\If{\texttt{$\Delta E$ < 0 or r < $e^{-\beta\Delta E}$}}{
				\texttt{Accept the flip}\\
				\texttt{Update E and M}\\
			}
			\texttt{Update the mean values for E and M}
		}
	}
	\texttt{Normalize all computed values}
\end{algorithm}

\section{Results}	% 0

\section{Discussion} %0

\section{Conclusion} %0

\bibliography{bib_proj4}
\bibliographystyle{plain}


\end{document}
